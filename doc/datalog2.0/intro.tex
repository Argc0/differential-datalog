\section{Introduction}\label{sec-introduction}

\subsection{Motivation}

Many real-world applications must update their output in response to input changes.  Consider, for
example, a cluster management system such as Kubernetes~\cite{kubernetes} that configures cluster
nodes to execute a user-defined workload.  As the workload changes, e.g., container instances are
added or removed from the system, the configuration must change accordingly.  In a large cluster,
computing configuration from scratch is prohibitively expensive.  Instead, modern cluster management
systems, including Kubernetes, apply changes incrementally, only updating state effected by the
change.

As another example, program analysis frameworks like Doop~\cite{Bravenboer-oopsla09} evaluate a set
of rules defined over the abstract syntax tree of the program.  Such an analyzer can be integrated
into the IDE to alert the developer as soon as a potential bug is introduced in the program.  This
requires re-evaluating the rules after every few keystrokes.  In order to achieve interactive
performance when working with very large code bases, the re-evaluation must occur incrementally,
preserving as much as possible intermediate results computed at earlier iterations.

Incremental algorithms tend to be significantly more complex than
their non-incremental versions.  In a nutshell, an incremental
algorithm propagates input changes to the output via all intermediate
steps.  This, in turn, requires (1) maintaining intermediate
computation results for each step, and (2) implementing an incremental
version of each operation, which, given an update to its input,
computes an update to its output.  Incremental computations that
operate on relational state are ubiquitous throughout systems
management software stacks.  The complexity of these algorithms
greatly impacts the development cost, feature velocity,
maintainability, and the performance of the control systems.

\emph{We argue that, instead of dealing with the complexity of
  incremental computation on a case-by-case basis,
  developers should embrace programming tools that solve the problem
  once and for all.}  In this paper we present one such tool -- \emph{Differential Datalog
  (DDlog)}~\cite{ddlog} -- a programming language that automates
incremental computation.  A DDlog programmer only has to write a
Datalog specification for the original (non-incremental) problem.  Given
this description, the DDlog compiler generates an efficient
incremental implementation.  

\subsection{Overview}

DDlog is a \emph{bottom-up}, \emph{incremental}, \emph{in-memory}, \emph{typed} Datalog engine for
writing \emph{embedded} deductive databases.

\textbf{Bottom-up:} DDlog starts from a set of ground facts (provided
by the user) and computes all possible derived facts by following
Datalog rules, in a bottom-up fashion.  (In contrast, top-down engines
are optimized to answer individual user queries without computing all
possible facts ahead of time.)  The bottom-up approach is preferable
in applications where all derived facts must be computed ahead of time
and in applications where the cost of initial computation is amortized
across a large number of queries.

\textbf{Incremental:} whenever the set of ground facts changes, DDlog
only performs the minimum computation necessary to compute all changes
in the derived facts.  This has significant performance benefits, and
only produces output of minimum size, minimizing communication
requirements.

\textbf{In-memory:} DDlog stores and processes data in
memory\footnote{In a typical use case, a DDlog program is used in
  conjunction with a persistent database; database records are fed to
  DDlog as inputs and the derived facts computed by DDlog are written
  back to the database; DDlog does not include a storage engine.}.  At
the moment, DDlog keeps all the data in the memory of a single
machine.  (This may change in the future, as DDlog builds on the
differential dataflow library~\cite{dd} that supports distributed
computation over partitioned data).

\textbf{Typed:} Pure Datalog does not have concepts like data types,
arithmetics, strings or functions.  To facilitate writing of safe,
clear, and concise code, DDlog extends Datalog with:
\begin{itemize}
\item A powerful type system, including Booleans, unlimited
  precision integers, bitvectors, strings, tuples, and
  Haskell-style tagged unions (but without recursive types).
  
\item The ability to store and manipulate sets, vectors, and maps as
  first-class values in relations, including performing aggregations.
  
\item Standard integer and bitvector arithmetic.
  
\item A simple procedural language that allows expressing many
  computations over these datatypes in DDlog without resorting to
  external functions.
  
\item String operations, including string concatenation and
  interpolation.
\end{itemize}

\textbf{Embedded:} while DDlog programs can be run interactively via a
command line interface, the primary use case is to run DDlog in the
same address space with an application that requires deductive
database functionality.  A DDlog program is compiled into a Rust
library that can be linked against a Rust, C/C++ or Java program
(bindings for other languages can be easily added).

DDlog is an open-source project, hosted on github~\cite{ddlog}, with a permissive MIT-license.
